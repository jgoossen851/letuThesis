\usepackage{tikz}
\usepackage{nccmath} % Extends asmmath
% Use Times New Roman-like font, from letuThesis.dtx
\usepackage[largesc]{newtxtext}
\usepackage[varg]{newtxmath}

\usetikzlibrary{shapes, shadows, arrows, matrix, chains, scopes, calc, shadows.blur}
\usetikzlibrary{fit, patterns, decorations.markings} % multithreading diagrams
\usetikzlibrary{decorations.pathreplacing} % LMS Double Buffer
\usetikzlibrary{shapes.misc} % for cross in Upsampling Scheme Graphic

% Shortcuts and user defined commands
% =======================

% \arrow: Draw an arrow from #1 to #3, connecting with symbol in #2
\newcommand\arrow[3]{\draw[->] (#1) #2 (#3);}

% \multiline: Format the #1 text in a box to break lines in fixed-size node
\newcommand\multiline[1]{\parbox{\textwidth}{\centering#1}}

% \delayby: Formatting for z-transform text
\newcommand\delayby[1]{$z^{-#1}$}


% Structure Commands to make new nodes or text as certain positions
% ==========================================

% \maketwoinputs: Create two notes on the left of #1 (named with suffix `-in1' and `-in2') equally spaced from center
\newcommand\maketwoinputs[1]{
    \path (#1.west) -- (#1.north west) coordinate[pos=0.5] (#1-in1);
    \path (#1.west) -- (#1.south west) coordinate[pos=0.5] (#1-in2);
}

% \maketwooutputs: Create two nodes on the right of #1 (named with suffix `-out1' and `-out2') equally spaced from center
\newcommand\maketwooutputs[1]{
    \path (#1.east) -- (#1.north east) coordinate[pos=0.5] (#1-out1);
    \path (#1.east) -- (#1.south east) coordinate[pos=0.5] (#1-out2);
}

% \outlabel: Create a label #3 above the output of a node #2 with optional shift #1
\newcommand\outlabel[3][(0,0)]{\node at ($ (#2.east) + #1 $) [anchor=south west] {#3};}

% \inlabel: Create a label #3 above the input of a node #2 with optional shift #1
\newcommand\inlabel[3][(0,0)]{\node at ($ (#2.west) + #1 $) [anchor=south east] {#3};}

% \abovelabel: Create a label #3 above the node #2 with optional shift #1
\newcommand\abovelabel[3][(0,0)]{\node at ($ (#2.north) + #1 $) [anchor=south] {#3};}

% \wrappernode: Draw large node #2, at (#3), with width #4 and height #5 [extra styles #1] and label #6
\newcommand\wrappernode[7][]{
    \node (#2) [block, #1, minimum width=#4, minimum height=#5] at (#3) {};
    \node at ($(#2.north west)+(3mm,0)$) [anchor=north west, blockfont, minimum width=12mm,
         blockheight, text width=#7] {\hspace{2mm}{\parbox{\textwidth}{#6}}};
}

% \mux: create #1 output nodes (suffix `-out[1:#1]') on base node #2 at coordinate #3
\newlength{\muxscale}\setlength{\muxscale}{0.75cm}
\newcommand\mux[3]{
    \node (#2) at (#3) [rectangle, minimum width=.28\muxscale, 
        minimum height=#1\muxscale, anchor=center] {}; % base node
    \foreach \y in {1,...,#1}{
        \node (#2-out\y) at ($(#2.north) + (0,0.5\muxscale) - (0,\y\muxscale)$)
          	 [rectangle, minimum width=0.28\muxscale, 
          	 minimum height=1\muxscale, anchor=center] {}; % output node
    }
    \filldraw [fill=black!50, scale=0.5, draw=black] (#2.west)
        -- ++($(0,#1\muxscale)+(0,-0.4\muxscale)$) -- ++(0.6\muxscale,0.4\muxscale)
        -- ++($(0,-#1\muxscale)+(0,-#1\muxscale)$) -- ++(-0.6\muxscale,0.4\muxscale)
        -- cycle; % displayed shape
}

% \demux: create #1 output nodes (suffix `-out[1:#1]') on base node #2 at coordinate #3 with width #4, output height #5
\newcommand\demux[5]{
    \newlength{\muxwidth}\setlength{\muxwidth}{#4}
    \newlength{\muxheight}\setlength{\muxheight}{#5}
    \node (#2) at (#3) [rectangle, minimum width=\muxwidth, 
        minimum height=#1\muxheight, anchor=center] {}; % base node
    \foreach \y in {1,...,#1}{
        \node (#2-out\y) at ($(#2.north) + (0,0.5\muxheight) - (0,\y\muxheight)$)
          	 [rectangle, minimum width=\muxwidth, 
          	 minimum height=1\muxheight, anchor=center] {}; % output node
    }
    \filldraw [fill=black!50, scale=0.5, draw=black] (#2.west)
        -- ++($(0,#1\muxheight)+(0,-0.4\muxheight)$) -- ++(2.5\muxwidth,0.4\muxheight)
        -- ++($(0,-#1\muxheight)+(0,-#1\muxheight)$) -- ++(-2.5\muxwidth,0.4\muxheight)
        -- cycle; % displayed shape
}

% Multithreading diagram colors
\colorlet{mainthreadcolor}{gray!70!white}
\colorlet{nullcolor}{red!20!white}
\colorlet{commandcolor}{gray}
\colorlet{computecolor}{blue!20!white!80!green}

% Fading for the multithreading diagrams (89% of full width)
\begin{tikzfadingfrompicture}[name=myfadewest]
  \clip (0,0) rectangle (2,2); % 2cm x 2cm fade mask
  \shade [left color=transparent!100, right color=transparent!100]
  (0,0) rectangle (0.22,2);
  \shade [left color=transparent!100, right color=transparent!0]
  (0.22,0) rectangle (2,2);
\end{tikzfadingfrompicture}
\begin{tikzfadingfrompicture}[name=myfadeeast]
  \clip (0,0) rectangle (2,2); % 2cm x 2cm fade mask
  \shade [left color=transparent!0, right color=transparent!100]
  (0,0) rectangle (1.78,2);
  \shade [left color=transparent!100, right color=transparent!100]
  (1.78,0) rectangle (2,2);
\end{tikzfadingfrompicture}
\begin{tikzfadingfrompicture}[name=myappendfadewest]
  \clip (0,0) rectangle (2,2); % 2cm x 2cm fade mask
  \shade [left color=transparent!100, right color=transparent!100]
  (0,0) rectangle (0.22,2);
  \shade [left color=transparent!100, right color=transparent!0]
  (0.22,0) rectangle (1.65,2);
  \shade [left color=transparent!0, right color=transparent!0]
  (1.65,0) rectangle (1.78,2);
  \shade [left color=transparent!100, right color=transparent!100]
  (1.78,0) rectangle (2,2);
\end{tikzfadingfrompicture}
\begin{tikzfadingfrompicture}[name=myappendfadeeast]
  \clip (0,0) rectangle (2,2); % 2cm x 2cm fade mask
  \shade [left color=transparent!100, right color=transparent!100]
  (0,0) rectangle (0.22,2);
  \shade [left color=transparent!0, right color=transparent!0]
  (0.22,0) rectangle (0.35,2);
  \shade [left color=transparent!0, right color=transparent!100]
  (0.35,0) rectangle (1.78,2);
  \shade [left color=transparent!100, right color=transparent!100]
  (1.78,0) rectangle (2,2);
\end{tikzfadingfrompicture}

% Parameters for the line-break symbol in the multithreading diagram
\def\MarkLt{4pt}
\def\MarkSep{4pt}

%%%%%%%%%%%%%
% Tikz environment setup %
%%%%%%%%%%%%%

% Options for the node styles, labels, lines, etc.
% ===========================
\tikzset{
    node distance=5mm, auto,
    >=latex, % Set arrow tip to LaTeX style
    % Globally increase line width
    every picture/.style = {
        line width=1 pt
    },
    % Global color style
    blockcolors/.style = {
        thick, draw=black,
        fill=gray!30!white,
        blockfont,
        drop shadow={color=gray!50!black},
    },
    % Global Block-Diagram Font
    blockfont/.style = {
       font=\small
    },
    % Generic Block Height
    blockheight/.style = {
        minimum height=10mm
    },
    % Basic Rectangular Block
    block/.style={
        rectangle, minimum size=6mm, minimum width=12mm,
        blockheight,
        node distance=5mm,
        blockcolors,
    },
    % Circle Node
    open circle/.style={
        circle, inner sep=0pt,
        thick, draw=black,
        fill = white,
    },
    % Digital Delay (z-transform)
    delay/.style={
        block, % copy the syle of block, but with modifications below
        rectangle, minimum size=4mm, minimum width=6mm,
        minimum height=6mm,
    },
    % System Input
    input/.style={
        open circle, minimum size=2mm, node distance=8mm, fill=gray!40!white
    },
    % System Output
    output/.style={
        input
    },
    % Regular Triangular Gain
    gain/.style={
       draw,
       shape border rotate=-90,
       inner sep=0.5mm,
       regular polygon,
       regular polygon sides=3,
       blockcolors,
       minimum size=1cm
    },
    % Isosceles Triangular Gain
    iso gain/.style={
       draw,
       isosceles triangle,
       inner sep=0.5mm,
       blockcolors,
       minimum size=1cm
    },
    % Summation Operator
    sum/.style={
        open circle, minimum size=7mm, node distance=8mm,
        blockcolors,
    },
    % Summation Operator (alternate)
    sum2/.style={
        open circle, minimum size=4mm, node distance=8mm,
        blockcolors, label={center:\large$+$},  %, label={\large$+$}
    },
    % Triangular Multiplication Operator
    mult/.style={
        gain,
        inner sep=-.65mm,
        font={\Large$\times$}
    },
    % Circular Multiplication Operator
    circle mult/.style={
        sum,
        font={\Large$\times$}
    },
    % Junction Marker on diverging lines
    junction/.style={
        open circle, minimum size=0.5mm,fill=black, node distance=5mm
    },
    % Node Location with no symbol
    phantom/.style={
    },
    %% Multithreading Diagram style
    null-line/.style = {
		rectangle, 
		draw=gray!60!black, thin,
		preaction={fill, nullcolor}, 
		pattern=north east lines,
		pattern color=red!70
	},
	calculation/.style = {
		rectangle,
		fill=computecolor,
		draw=black,
		anchor = west,
	},
	command/.style = {
		calculation,
		fill=commandcolor,
		inner sep=-0.5\pgflinewidth, outer sep=0,
	},
	cmddot/.style = {
		open circle,
		fill = black,
		minimum size = 1mm,
	},
	fadeout/.style = {
		command,
		path fading=myfadeeast,
		left color=commandcolor,
		right color=white,
	},
	fadein/.style = {
		command,
		path fading=myfadewest,
		left color=white,
		right color=commandcolor,
	},
	TwoMarks/.style={
		postaction={
    		decorate, decoration={
				markings, mark=at position #1 with
				{
					\begin{scope}[xslant=0.3]
					\draw[line width=\MarkSep,white,-] (0pt,-\MarkLt) -- (0pt,\MarkLt) ;
					\draw[-] (-0.5*\MarkSep,-\MarkLt) -- (-0.5*\MarkSep,\MarkLt) ;
					\draw[-] (0.5*\MarkSep,-\MarkLt) -- (0.5*\MarkSep,\MarkLt) ;
					\end{scope}
				}
			}
		}
	},
	TwoMarks/.default={0.5},
	cross/.style={
		cross out, draw=black, minimum size=2*(#1-\pgflinewidth),
		inner sep=0pt, outer sep=0pt
	},
	cross/.default={3pt}, %default radius will be 3pt. 
}

