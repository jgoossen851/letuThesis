% \iffalse meta-comment
%
% Copyright (C) 2020 by Jeremy Goossen
%
% This file may be distributed and/or modified under the
% conditions of the LaTeX Project Public License, either
% version 1.3 of this license or (at your option) any later
% version. The latest version of this license is in:
%
% http://www.latex-project.org/lppl.txt
%
% and version 1.3 or later is part of all distributions of
% LaTeX version 2005/12/01 or later.
%
% \fi
%
% \iffalse
%<*driver>
\ProvidesFile{letuThesis.dtx}
%</driver>
%
%<package>\NeedsTeXFormat{LaTeX2e}[2005/12/01]
%<package>\ProvidesPackage{letuThesis}
%<*package>
	[2019/07/27 v1.0 Thesis Styling for LeTourneau University School of Engineering]
%</package>
%
%<*batchfile>
\begingroup

\input docstrip.tex
\keepsilent

\preamble

This is a generated file.

Copyright (C) 2020 by Jeremy Goossen

This file may be distributed and/or modified under the
conditions of the LaTeX Project Public License, either
version 1.3 of this license or (at your option) any later 
version. The latest version of this license is in:

	http://www.latex-project.org/lppl.txt
	
and version 1.3 or later is part of all distributions of 
LaTeX version 2005/12/01 or later.

\endpreamble

\askforoverwritefalse
\generate{\file{letuThesis.sty}{\from{letuThesis.dtx}{package}}}



\endgroup
%</batchfile>
%
%<*driver>
\documentclass{ltxdoc}
\EnableCrossrefs
\CodelineIndex
\RecordChanges
%\OnlyDescription
%% Packages/Setup for compiling the .dtx documentation
\usepackage{xspace}
\newcommand\toc{\textsc{toc}\xspace}
\newcommand\lot{\textsc{lot}\xspace}
\newcommand\lof{\textsc{lof}\xspace}
\begin{document}
	\DocInput{letuThesis.dtx}
	\PrintChanges
	\PrintIndex
\end{document}
%</driver>
% \fi
%
% \CheckSum{0}
%
% \CharacterTable
% {Upper-case \A\B\C\D\E\F\G\H\I\J\K\L\M\N\O\P\Q\R\S\T\U\V\W\X\Y\Z
% Lower-case \a\b\c\d\e\f\g\h\i\j\k\l\m\n\o\p\q\r\s\t\u\v\w\x\y\z
% Digits \0\1\2\3\4\5\6\7\8\9
% Exclamation \!       Double quote \"     Hash (number) \#
% Dollar \$            Percent \%          Ampersand \&
% Acute accent \'      Left paren \(       Right paren \)
% Asterisk \*          Plus \+             Comma \,
% Minus \-             Point \.            Solidus \/
% Colon \:             Semicolon \;        Less than \<%
% Equals \=            Greater than \>     Question mark \?
% Commercial at \@     Left bracket \[     Backslash \\
% Right bracket \]     Circumflex \^       Underscore \_
% Grave accent \`      Left brace \{       Vertical bar \|
% Right brace \}       Tilde \~}
%
% \changes{v1.0}{2019/07/27}{Initial version}
%
% \GetFileInfo{letuThesis.dtx}
%
% \DoNotIndex{\newenvironment}
% \DoNotIndex{\par,\gdef,\global,\let,\relax,\centering,\copyright,\fi,\quad}
%
% \overfullrule=40pt
% \setcounter{IndexColumns}{2}
%
% \title{The \textsf{letuThesis} package\thanks{This document
% corresponds to \textsf{letuThesis.dtx}~\fileversion, dated~\filedate.}}
% \author{Jeremy Goossen \\ \texttt{jeremyg995@gmail.com}}
%
% \maketitle
%
% \section{Introduction}
% This class formats a \LaTeX{} document according to LeTourneau University's
% guidelines for Master's Thesis papers.
% This package correlates to version 1.4 of the University's guidelines,
% published September 2019.
%

%
% \section{Usage}
%
% \subsection{Preamble Statements}
%
% \subsubsection{Recommended Fields}
%
% To set the fields needed for the preliminary pages, the following macros are
% called in the preamble (before |\begin{document}|) with a single argument
% describing their desired value.
% 
% \DescribeMacro{\title}
% This macro sets the title of the thesis. If line-breaks are desired in the title on
% the title page, they can be entered using the macros, |\\|, |\break|, |\par|,
% or |\newline|. The title appears on the title page, on the signature page, and
% with all line-breaks removed on the abstract.
% Default: {\itshape How to Prepare \textbackslash \textbackslash\ 
% the Perfect Thesis Document}.
%
% Note: All line-breaks are converted to |\break| with string replacement following
% a single expansion. Therefore, if a |\\|, |\break|, or |\newline| is saved as a new
% command used in the title, the syntax will break. Saving a |\par| as a new command
% and entering it into the title will create a new line on the title page, signature page,
% and the abstract.
%
% \DescribeMacro{\author}
% This macro sets the author name for the title page, signature page,
% copyright page, and abstract.
% If necessary, the names on the title page and signature
% page can be given a different format by defining alternate text using the
% |\titlepageauthor| and |\signaturepageauthor| macros described below.
% Default: {\itshape Firstname Lastname}
%
% \DescribeMacro{\committee}
% This macro lists out the committee members on the signature page, separated
% by the |\\| symbol, the standard symbol for splitting rows of tables.
% The first member listed will be given
% the label chair, and that member's name (without suffixes) will be used as the default
% value of the Director (see below).
% Default: {\itshape Dr. William Graph, PhD |\\| Dr. Jane Doe, PhD, PE |\\| Dr. Doe Do, PhD}
% 
% \DescribeMacro{\date}
% This macro sets the date of the paper. Only the year is printed on the title page
% and the copyright page. Since the year is extracted as the last four characters
% of the expansion of the date, the format of the date must keep the year at the end.
% Default: {\itshape |\today|}
%
% \DescribeMacro{\university}
% This macro sets the university for the title page and signature page. 
% It should be in sentence case, as it is printed on the bottom of the title page too.
% Default: {\itshape LeTourneau University}
%
% \DescribeMacro{\school}
% This macro sets the school, which is typeset below the university. 
% It will automatically be converted to Uppercase by \LaTeX{}. 
% Default: {\itshape School of Engineering and Engineering Technology}
%
% \DescribeMacro{\discipline}
% This macro describes the field of study.
% It appears on the title page in the text, ``in the Discipline of \meta{discipline},'' 
% on the signature page, ``APPROVED FOR THE DISCIPLINE OF \meta{DISCIPLINE},'' 
% and on the Abstract, ``Master of Science in \meta{discipline}.'' 
% Default: {\itshape Underwater Basket Weaving}
%
% \DescribeMacro{\lastChapter}
% This macro contains the chapter number and name of the last chapter as it 
% should appear in the Abstract after the page count.
% No structure is implemented currently to automatically supply the chapter of the
% last page; therefore, the field defaults to a warning.
% Default: {\itshape \textbf{Last Chapter Not Specified}}
%
% \subsubsection{Fine-Tuning Fields}
%
% The following fields are not necessary to typeset an appropriate set of preliminary pages,
% but they may be needed to specify a different format when the automatic formatting
% done by \LaTeX{} is incorrect or does not meet the template.
%
% \DescribeMacro{\director}
% This macro allows overwriting the thesis director's name in the Abstract.
% The default value of the director is set as the text of first committee member before any
% commas. If a comma appears in the director's name on the abstract or a different name
% should be displayed, that name should be entered using this macro to explicitly set
% the thesis director name.
%
% \DescribeMacro{\titlepageauthor}
% This macro allows the author to style the
% author name uniquely for the title page.
% If this macro is not used, the name on the
% title page will default to the author name.
%
% \DescribeMacro{\signaturepageauthor}
% This macro allows the author to style the
% author name uniquely for the signature page.
% If this macro is not used, the name on the
% signature page will default to the author name.
%
% \DescribeMacro{\centersectionwithnumber}
% By default, section numbers, if enabled, do not affect the centering
% of the section title on the page. The centered section title text simply
% has the number placed to its left, such that the title will be typeset in
% exactly the same place whether or not section numbers are enabled.
% To disable this feature such that the title is centered with respect to
% both the section number and text, use the macro
% |\centersectionwithnumber| in the preamble.
%
% \DescribeMacro{\nosectionnumbers}
% This macro disables section numbering in the body of the text.
%
% \DescribeMacro{\nosubsectionnumbers}
% This macro disables sub-section numbering in the body of the text.
%
% \DescribeMacro{\setsubsubsecprestretch}
% Call |\setsubsubsecprestretch|\marg{number} to change the
% vertical separation between the body text and the beginning of a
% new subsubsection. The stretch value \meta{number} is applied
% as a stretch factor for the |\baselineskip| in addition to the
% |\baselineskip| from a new paragraph. Therefore, a value of
% \meta{number}=0 will add no extra space between the text and
% the new subsubsection, while \meta{number}=1 adds a full
% empty line between them. Default: \emph{0.5}.
%
% \DescribeMacro{\setcontentslabelwidth}
% By default, 0.4 inches of space is reserved on the left for the
% numbered labels of the List of Figures and List of Tables.
% This space can be overwritten using the
% |\setcontentslabelwidth|\marg{length} command in the preamble
% after the |letuThesis| package has been loaded.
% The single mandatory length argument must include units.
%
% \subsection{Sectioning Commands}
%
% Several commands are provided to split the thesis document into several sections
% that have different formating, similar to the commands found in the |book|
% documentclass.
% 
% \DescribeMacro{\frontmatter}
% The |\frontmatter| macro should be called after the document has begun, but before
% any text is typeset. The macro ensures that a new page will be started and restarts
% page numbering with roman numerals on the following page.
%
% \DescribeMacro{\mainmatter}
% The |\mainmatter| macro separates the preliminary pages from the main content
% of the document. The macro ensures a new page will be started and restarts
% page numbering with Arabic numerals. The macro also changes the format of the
% chapter titles to be bold-face.
%
% \DescribeMacro{\backmatter}
% The |\backmatter| macro changes the format of chapter titles to be regular text
% so that unnumbered chapters after the appendices are formatted consistently with
% chapters before the first.
%
%
% \subsection{Preliminary Pages}
%
% Several commands are provided to automatically format the required preliminary
% pages for the thesis.
%
% \DescribeMacro{\maketitle}
% The |\maketitle| macro typesets the title page, making use of the macros
% set by the |\university|, |\school|, |\title|, |\author|, |\discipline|, and |\date|
% macros described above. If the author is not automatically typeset as desired,
% the automatically generated text can be overwritten for this page only
% using the |\titlepageauthor| macro.
%
% \DescribeMacro{\makesignatures}
% The |\makesignatures| macro typesets the signature page, making use of the macros
% set by the |\university|, |\school|, |\title|, |\author|, |\discipline|, and |\committee|
% macros described above.
% Signature lines are generated for each committee member specified, with the chair
% listed first. If the author is not automatically typeset as desired,
% the automatically generated text can be overwritten for this page only
% using the |\signaturepageauthor| macro.
%
% \DescribeMacro{\makecopyright}
% The |\makecopyright| macro typesets the copyright statement page,
% making use of the macros set by the |\date| and |\author|
% macros described above.
%
% \DescribeMacro{\makeabstract}
% The |\makeabstract| macro typesets the abstract page, making use of the macros
% set by the |\author|, |\discipline|, |\committee|, and |\lastChapter|
% macros described above. The macro accepts one argument containing
% the text of the abstract. If the thesis director is not automatically typeset as desired,
% the automatically generated text can be overwritten for this page only
% using the |\director| macro.
%
% \DescribeMacro{\makeacknowledgments}
% The |\makeacknowledgments| macro typesets the acknowledgments page.
%
%
% \section{Environments}
% \DescribeEnv{thesistable}
% Tables for the thesis are given a specific environment to enable easy access
% to good-looking and consistent formatting. Table design is not specified in the
% LeTourneau University Thesis guidelines, so alternative designs may be used
% by loading separate packages.
% The |thesistable| environment can either typeset a floating table or a multi-page
% ``longtable'' at the current position in the text. The environment takes six arguments,
% some of which are optional, as described below:
% \begin{enumerate}
% \item \oarg{placementSpecifier} The first argument is optional and contains
% the desired float position of the table. \textbf{If provided}, the table will be typeset as
% a floating table and must fit on a single page. The \meta{placementSpecifier}
% field should be in the same format as would be provided to any other float,
% e.g., |[!htb]|. The table will by typeset using the underlying |tabular| and |table| environments.
% \textbf{If omitted}, the table will be typeset at the current position in the text, but
% can span multiple pages. The table will be typeset using the underlying |longtable|
% environment from the |longtable.sty| package.
% \item \marg{columnAlignment} The second argument contains the column format,
% specified in the same way as it would be for the |tabular| environment. Each column
% desired in the table should be represented using a letter to encode the desired text
% alignment within the column, e.g., |l| for left justified, |r| for right-justified,
% |c| for centered, or |p{|\meta{width}|}| for a paragraph with line breaks. Other,
% more complex options may also be used---see the documentation for the 
% |tabular| environment.
% \item \oarg{lofCaption} The third and fourth arguments contain the caption text for the
% table. The third argument is optional. If provided, the \meta{lofCaption} caption will
% be used to reference the table in the List of Tables instead of the \meta{caption}
% that is located under the table in the body of the text. This optional argument is useful
% when the caption is too long to fit in the List of Tables and should be abbreviated.
% This option is also useful if the caption needs special formatting to be typeset correctly
% in either location.
% \item \marg{caption} The fourth argument will be used as the caption for the table.
% If the third argument, \meta{lofCaption}, is not provided, the \meta{caption}
% argument will also be used to refer to the table in the List of Tables.
% \item \marg{label} The fifth argument is the table label for reference to the
% table elsewhere in the document. Using the macro |\ref{|\meta{label}|}| in the text
% will typeset this table's number.
% \item \marg{titleRow} The sixth and final argument contains the entire title row of
% the table. The title, or header, row is formatted differently than the body of the table.
% As \meta{titleRow} is a part of the table, it must have the same format as any other 
% row in the table, with cells separated by the |&| character. However, the
% \meta{titleRow} should not have the final newline directive (|\\|). 
% \end{enumerate}
% The contents of the table should go between the |\begin| and |\end| directives
% of the environment, formatted with the row and column separators
% (|&| and |\\| respectively) like any other table.
% The last row of the table must contain the column separator (|\\|) to avoid errors.
%
%
% \section{Known Bugs}
% If section numbers are left turned on in the body of the document and a section
% label take almost the whole line or more than a line, the section number can
% be pushed outside of the page margins. The section number is typeset as if it had
% no width in order to preserve the centered look of the text itself. Line breaks should
% manually be added to the title.
%
% Upon loading a new version of MiKTeX on another computer,
% my \LaTeX{} documents would no longer compile.
% It seems that the fonts were not being loaded right,
% and pdfLaTeX was not able to typeset the document.
% I was getting the error message, ``Font ec-qtmr at 480 not found.''
% This problem was fixed by following the steps as laid out by ltcomdata
% and the comment by CarLaTeX on Stack Exchange:
% https://tex.stackexchange.com/questions/280631/
% new-font-file-not-found-with-miktex-installation.
% Open the command prompt with administrator privileges.
% Then type the following commands in order:
%
% |updmap --admin|
%
% |initexmf --admin --mkmaps|
%
% |initexmf --mkmaps|
%
% The |\makeabstract| macro automatically provides a word-count
% of the abstract text. If the text of the abstract is not right against
% the surrounding brackets, extra words will be added to the
% word-count. Either place the brackets right next to the abstract text,
% or comment out the newline characters at the end of the line,
% as shown in the example below:
%
% |\makeabstract{%|
% 
% |The body text of the abstract.%|
% 
% |}|
%
% \section{To Do}
% Some of these have already been implemented, other still have not.
% These items came from the lab white-board where they were
% originally brainstormed.
% \begin{itemize}
% \item Ensure that the format for all commands [|\macro|]
% is consistent.
% \item Add and if/then construct to disable the reduced margin
% sizes in the |\section| headings.
% \end{itemize}
%
% \StopEventually{\PrintChanges}
%
%%%%%%%%%%%%%%%%%%%%%%%%%%%%%%%%%%%%%%%%
%
% \section{Implementation}
%
% \subsection{Basic Geometry and Typesetting}
%
% Load the packages needed for formatting. The |geometry| package
% provides options for setting the margins. The package is loaded with
% options specifying that all margins should be 1 inch,
% while the left margin is 1.5 inches to allow for binding. 
% The page numbers (in the footer) are placed 0.5 inches
% from the body text.
% The |setspace| package is also loaded
% for double-spacing the document.
% The |indentfirst| package ensures that the first paragraph
% in a section or chapter is also indented,
% which is false in a default \LaTeX{} document.
% Finally, the extra space between the top of the printable page
% and the text is removed.
%    \begin{macrocode}
\RequirePackage[margin=1in, left=1.5in, footskip=0.5in]{geometry}
\RequirePackage{setspace}
\RequirePackage{indentfirst}
\topskip=0pt
%    \end{macrocode}
%
% Include libraries to load the |newtxtext| and |newtxmath| fonts,
% based on the \TeX{} Gyre Termes font,
% in turn based on Times New Roman.
% Set the size of a paragraph indent to be 0.5 inches.
% Set the spacing using the command |\doublespaced|
% for the entire document. If needed, the commands
% |\onehalfspacing| and |\singlespacing| are available for portions
% of the document that should not be double spaced.
% (A base \LaTeX{} command to achieve double spacing is
% |\linespread{1.6}|, though this command did not seem to
% give as nice results.)
%    \begin{macrocode}
\RequirePackage[largesc]{newtxtext}
\RequirePackage[varg]{newtxmath}
%\RequirePackage[T1]{fontenc}
\setlength{\parindent}{0.5in}
\doublespacing
%    \end{macrocode}
%
% Set values for the penalties for widow and orphan lines.
% Values of 10000 indicate infinite badness, so \LaTeX{} will avoid
% widows and orphans if at all possible.
%    \begin{macrocode}
\widowpenalty10000
\clubpenalty10000
%    \end{macrocode}
%
%
% \subsection{Preamble Fields}
%
% In order to process strings in the arguments of the preamble fields and,
% later on, extensively with formatting the preliminary pages,
% the |xstring| package is used.
%    \begin{macrocode}
\RequirePackage{xstring}
%    \end{macrocode}
%
% \subsubsection{Recommended Fields}
% For each of the commands used to enter information in the
% preamble (with noted exceptions), a command is created to enter
% the information followed by the definition of a default value.
% No new command is created for the \meta{author} since the
% |report| class already provides an |\author| command.
% Likewise, the |report| class already defines the |\date| command
% and the default |\@date| macro, so they are not redefined here.
% The commands |\title| and |\author| are redefined while the rest
% of the fields are new commands provided by this package.
%
% Default values are specified with the |\providecommand| macro to
% allow different defaults to be specified before this package is loaded.
%
% \begin{macro}{\title}
% Only the |\title| command edits the formatting of the argument
% supplied before saving it in its global content macro. 
% \DescribeMacro{\expandarg}
% First, |\expandarg| is called to expand the first string in each
% argument to the string-processing commands from the |xstring|
% package. To keep a macro from being expanded, an |\empty|
% macro is appended beforehand to be expanded to nothing first.
% The first three lines replace any |\par|s, |\newline|s, and |\\|s
% in the supplied title with a |\break| macro, which is the only one
% that can be typeset correctly.
% Then, any spaces before a |\break| macro are removed to ensure
% that each line of the title is truly centered, not slightly shifted to the
% left by a trailing space on the line.
% Finally, set the default title, |\@title|, to be used if the |\title|
% command is never issued in the document.
%    \begin{macrocode}
\renewcommand{\title}[1]{%
  \expandarg
  \StrSubstitute{#1}{\empty\par}{\empty\break}[\@title]
  \StrSubstitute{\@title}{\empty\newline}{\empty\break}[\@title]
  \StrSubstitute{\@title}{\empty\\}{\empty\break}[\@title]
  \global\StrSubstitute{\@title}{\space\break}{\empty\break}[\@title]
}
\renewcommand*\@title{How to Prepare\break the Perfect Thesis Document}
%    \end{macrocode}
% \end{macro}
%
% \begin{macro}{\author}
% Set the |\@author| macro default value.
%    \begin{macrocode}
\renewcommand*\@author{Firstname Lastname}
%    \end{macrocode}
% \end{macro}
%
% \begin{macro}{\committee}
% Define the |\committee| command and set the |\@committee|
% macro default value.
%    \begin{macrocode}
\newcommand{\committee}[1]{\gdef\@committee{#1}}
\providecommand*\@committee{%
  Dr. William Graph, PhD \\ Dr. Jane Doe, PhD, PE \\ Dr. Doe Do, PhD
}
%    \end{macrocode}
% \end{macro}
%
% \begin{macro}{\university}
% Define the |\university| command and set the |\@university|
% macro default value.
%    \begin{macrocode}
\newcommand{\university}[1]{\gdef\@university{#1}}
\providecommand*\@university{LeTourneau University}
%    \end{macrocode}
% \end{macro}
%
% \begin{macro}{\school}
% Define the |\school| command and set the |\@school|
% macro default value.
%    \begin{macrocode}
\newcommand{\school}[1]{\gdef\@school{#1}}
\providecommand*\@school{%
  School of Engineering and Engineering Technology
}
%    \end{macrocode}
% \end{macro}
%
% \begin{macro}{\discipline}
% Define the |\discipline| command and set the |\@discipline|
% macro default value.
%    \begin{macrocode}
\newcommand{\discipline}[1]{\gdef\@discipline{#1}}
\providecommand*\@discipline{Underwater Basket Weaving}
%    \end{macrocode}
% \end{macro}
%
% \begin{macro}{\lastChapter}
% Define the |\lastChapter| command and set the |\@lastChapter|
% macro default value.
%    \begin{macrocode}
\newcommand{\lastChapter}[1]{\gdef\@lastChapter{#1}}
\providecommand*\@lastChapter{\textbf{Last Chapter Not Specified}}
%    \end{macrocode}
% \end{macro}
%
%
% \subsubsection{Fine Tuning Fields}
% The optional fields do not provide a default value since their value
% is set automatically by \LaTeX{} if the macro is not used.
% The definitions are set here.
%
% \begin{macro}{\director}
%    \begin{macrocode}
\newcommand{\director}[1]{\gdef\@director{#1}}
%    \end{macrocode}
% \end{macro}
%
% \begin{macro}{\titlepageauthor}
%    \begin{macrocode}
\newcommand{\titlepageauthor}[1]{\gdef\@titlepageauthor{#1}}
%    \end{macrocode}
% \end{macro}
%
% \begin{macro}{\titlepageauthor}
%    \begin{macrocode}
\newcommand{\signaturepageauthor}[1]{\gdef\@signaturepageauthor{#1}}
%    \end{macrocode}
% \end{macro}
%
% \begin{macro}{\centersectionwithnumber}
% Define the default formatting of section numbers inside a left
% overlap macro. Then provide a new macro to overwrite the default
% definition to disable the left overwrite.
%    \begin{macrocode}
\newcommand\centersectionwithnumber{\gdef\sec@numformat{}}
\providecommand*\sec@numformat[1]{\llap{#1}}
%    \end{macrocode}
% \end{macro}
%
% \begin{macro}{\nosectionnumbers}
% \begin{macro}{\nosubsectionnumbers}
% Internal macros for the numbering of the section and subsection are
% enabled by default, but the commands
% |\nosectionnumbers| and |\nosubsectionnumbers|
% can overwrite the default behavior.
%    \begin{macrocode}
\newcommand{\nosectionnumbers}{\gdef\@sectionnumbering{}}
\providecommand*\@sectionnumbering{\sec@numformat{\thesection\quad}}
\newcommand{\nosubsectionnumbers}{\gdef\@subsectionnumbering{}}
\providecommand*\@subsectionnumbering{\thesubsection\quad}
%    \end{macrocode}
% \end{macro}
% \end{macro}
%
% \begin{macro}{\setsubsubsecprestretch}
% Set the default baseline stretch of 0.5 and create a command
% for the user to overwrite the default.
%    \begin{macrocode}
\newcommand{\setsubsubsecprestretch}[1]{%
  \gdef\subsubsection@stretch{#1}}
\providecommand*\subsubsection@stretch{0.5}
%    \end{macrocode}
% \end{macro}
%
%
% \begin{macro}{\setcontentslabelwidth}
% Define the default width for the labels in the List of Tables and List of Figures,
% and define a command to overwrite the default value.
%    \begin{macrocode}
\newlength{\contents@labelwidth}
\setlength{\contents@labelwidth}{0.4in}
\newcommand\setcontentslabelwidth[1]{%
  \setlength{\contents@labelwidth}{#1}}
%    \end{macrocode}
% \end{macro}
%
% \subsection{Preliminary Pages}
%
% Each of the preliminary pages with complex formatting has a
% command to typeset it:
% |\maketitle|, |\makesignatures|, |\makecopyright|,
% |\makeabstract|, and |\makeacknowledgments|.
%
% \subsubsection{Title Page}
% First, a |\def| is used to define formatting to be
% repeated on both the title and signatures page.
% Spacing within this command has no stretchable glue,
% so the text will be exactly duplicated on the two pages.
% Since the |\doublespacing| of the |setspace| package
% actually sets the |\baselinestrech| value to 1.665 in order to
% approximately double the whitespace between the lines,
% an empty line between the University and the School
% needs 2/1.665 = 1.2 |\baselineskip|s of space.
% Since 1 baseline is already added from the newline,
% and an extra 0.6 baseline is added from the formatting
% the chapter, an additional space of |-0.4\baselineskip|
% is placed between the University and School.
% A constant amount of space (5 picas) is added between
% the School and Title so that the text is exactly duplicated
% on the Title and Signature Pages.
%    \begin{macrocode}
\gdef\titlepage@text{%
	{\vspace{-0.4\baselineskip}\MakeUppercase{\@school}\par}%
    \vskip 5pc%
    {\MakeUppercase\@title\par}}
%    \end{macrocode}
%
% \begin{macro}{\maketitle}
% The |\maketitle| macro typesets the title page.
% A centered environment contains the content of the title page.
% Each field of text is contained within a group (|{...}|), and is
% separated by different amounts of vertical space,
% measured in picas (1 pica = 12 points).
% Spacing on the page other than those contained in the
% |\titlepage@text| macro are given a minimum space,
% measure in picas, with added vertical glue, |\vfill|,
% which \LaTeX{} can expand to optimally fill the page.
%
% The author name is typeset within a centered tabular environment.
% If not already explicitly defined, the
% \DescribeMacro{\@titlepageauthor}|\@titlepageauthor|
% is defined as the thesis author.
%    \begin{macrocode}
\renewcommand{\maketitle}{%
  \chapter*{\@university}
  \begin{center}%
    \titlepage@text%
    \vskip 3pc \vfill
    {\singlespacing by\par%
    \begin{tabular}[t]{c}
      \providecommand{\@titlepageauthor}{\@author}
      \@titlepageauthor
    \end{tabular}\par}
    \vskip 3pc \vfill
    {A thesis submitted in partial fulfillment of \break%
    the requirements for the degree of Master of Science \break%
    in the Discipline of \@discipline \par}
    \vskip 3pc
    {\@university\par}
    \vskip 2pc
    {\@getYear\@date \par}
    \vskip 4pc
  \end{center}\par
%    \end{macrocode}
%
% Finally, glue is added after the centered environment to allow
% \LaTeX{} to expand the empty space before the |\null|
% placeholder at the end of the page.
% The page is specified to contain no page number.
% Then the |\maketitle| and |\@titlepageauthor| macros are
% disabled to prevent accidental later use.
%    \begin{macrocode}
  \vfill\null
  \thispagestyle{empty}
  \global\let\maketitle\relax
  \global\let\@titlepageauthor\@empty
}
%    \end{macrocode}
% \end{macro}
%
% \subsubsection{Signatures Page}
% \begin{macro}{\makesignatures}
% The |\makesignatures| macro typesets the signature page.
% The beginning of the macro is identical to the title page to ensure that the school name
% and thesis title match on both pages.
%
% A centered environment contains the content of the signatures page.
% Each field of text is contained within a group (|{...}|), and is separated by
% different amounts of vertical space, measured in picas (1 pica = 12 points).
% Exact amounts of space are placed between the top of the page and the university
% name, and between the university name and the thesis title. Exact spacing allows
% the spacing to exactly duplicate the spacing on the title page.
% Other spacings on the page
% are given a minimum space, measure in picas, with added vertical glue, |\vfill|,
% which \LaTeX{} can expand to optimally fill the page.
%
% The author name is typeset within a centered tabular environment.
% If not already explicitly defined, the
% \DescribeMacro{\@signaturepageauthor}|\@signaturepageauthor| is defined
% as the thesis author.
%
% The signature lines are typeset using the internal macro
% \DescribeMacro{\@typesetCommittee}|\@typesetCommittee|, 
% described below.
%    \begin{macrocode}
\newcommand{\makesignatures}{%
  \chapter*{\@university}
  \begin{center}%
    \titlepage@text%
    \vskip 1pc
    {\singlespacing by\par%
    \begin{tabular}[t]{c}
      \providecommand{\@signaturepageauthor}{\@author}
      \@signaturepageauthor
    \end{tabular}\par}
    \vskip 1pc
    {\MakeUppercase{A thesis \break%
    approved for the discipline of \break%
    \@discipline} \par}
    \vskip 48pt minus 12pt \vfill
    {By Thesis Committee\par}
    {\@typesetCommittee\par}
  \end{center}\vskip -24pt \par\nopagebreak
%    \end{macrocode}
%
% Finally, glue is added after the centered environment to allow \LaTeX{} to expand the
% empty space before the |\null| placeholder at the end of the page.
% Then, various macros are disabled to prevent accidental later use.
%    \begin{macrocode}
  \vfill\null
  \global\let\makesignatures\relax
  \global\let\@committee\@empty
  \global\let\@university\@empty
  \global\let\@school\@empty
  \global\let\@signaturepageauthor\@empty
  \global\let\committee\relax
  \global\let\university\relax
  \global\let\school\relax
}
%    \end{macrocode}
% \end{macro}
%
% \subsubsection{Copyright Statement Page}
% \begin{macro}{\makecopyright}
% The |\makecopyright| macro typesets the copyright statement page.
% The copyright starts a new unnumbered chapter and adds the chapter to the
% Table of Contents. The |textcomp| package is used to typeset the copyright symbol
% with a good font.
%
% The header is centered on the top lines, and the text of the copyright statement
% is spaced in a paragraph below.
%    \begin{macrocode}
\RequirePackage{textcomp}
\newcommand{\makecopyright}{%
  \chapter*{Copyright Statement}
  \addcontentsline{toc}{chapter}{COPYRIGHT}
  \vskip 0.5pc
  {\centering Copyright \copyright{} \@getYear\@date\ by \@author\par}
  \vskip 2.5pc
  All rights reserved. No part of this publication may be reproduced,
  stored in a retrieval system, or transmitted, in any form or by any
  means (electronic, mechanical, photocopying, recording, or
  otherwise) without the prior written permission of the author.\par
  \vfill
%    \end{macrocode}
% If the |\makecopyright| macro was called, the |\makecopyright|, |\@getYear|, |\@date|,
% and |\date| macros are disabled to prevent accidental later use.
% Since the copyright section is optional to the thesis, these commands are
% disabled again by the |\makeabstract| macro described below.
%    \begin{macrocode}
  \global\let\makecopyright\relax
  \global\let\@getYear\@empty
  \global\let\@date\@empty
  \global\let\date\relax
}
%    \end{macrocode}
% \end{macro}
%
% \subsubsection{Abstract Page}
% \begin{macro}{\makeabstract}
% The |\makeabstract| macro typesets the Abstract page and takes one argument
% containing the abstract text. It uses the package |lastpage| to obtain the page
% number of the last page of the thesis.
% The abstract starts a new unnumbered chapter and adds the chapter to the
% Table of Contents.
%    \begin{macrocode}
\RequirePackage{lastpage}
\newcommand{\makeabstract}[2][]{%
  \chapter*{Abstract}
  \addcontentsline{toc}{chapter}{ABSTRACT}
%    \end{macrocode}
% 
% The words in the abstract can be automatically counted if no special
% characters are found in the abstract.
% The words in the abstract are counted by counting the spaces, |\space|, and
% paragraphs, |\par| in the abstract text.
% The number of spaces is saved to a counter and incremented by the number
% of paragraphs. Then the counter is incremented by one more since there will
% be one more word than the total of the counted spaces and paragraphs.
% First, the |\count@given| macro stores the value of the optional argument,
% containing the number of words to display. Then, If that field is empty,
% the auto-count function is skipped and the |\@abstractwordcount| macro
% is set to the value given. Otherwise, the auto-count function will run.
%    \begin{macrocode}
  \def\count@given{#1}
  \newcount\@abstractwordcount
  \ifx\count@given\empty
    \typeout{optional argument was omitted}
    \fullexpandarg
    \StrCount{#2}{\space}[\@countedSpaces]
    \StrCount{#2}{\par}[\@countedPars]
    \@abstractwordcount \@countedSpaces\relax
    \advance\@abstractwordcount \@countedPars\relax
    \advance\@abstractwordcount 1\relax
  \else
    \typeout{optional word count was given: `#1'}
    \@abstractwordcount #1\relax
  \fi
%    \end{macrocode}
%
% If not already explicitly defined, or successfully set by the |\@typesetMember| macro,
% \DescribeMacro{\@director}|\@director| is defined as \textbf{Director not found!}
% to warn the user to set the director using the |\director| macro.
%
% The title of the thesis is reformatted to remove any line-breaks by replacing |\break|s
% with spaces. Then the fully expanded title is interpreted as a token to consolidate
% consecutive spaces.
%    \begin{macrocode}
  \providecommand*{\@director}{\textbf{Director not found!}}
  \expandarg
  \StrSubstitute{\@title}{\empty\break}{\space}[\@title]
  \fullexpandarg
  \tokenize{\@title}{\@title}
%    \end{macrocode}
%
% The first lines of the abstract are not indented.
% Grouped commands form the centered word count at the beginning of the abstract text,
% and then the text of the abstract itself is typeset.
%    \begin{macrocode}  
  \noindent\@author\space (Master of Science in \@discipline)\par
  \noindent\@title\par
  \noindent Directed by \@director\par
  \noindent\pageref{LastPage} pp., \@lastChapter\par
  {\centering(\the\@abstractwordcount\space words)\par}
  #2\par
%    \end{macrocode}
%
% Remaining macros are disabled to prevent accidental later use, some of which may
% have already been disabled by the |\makecopyright| macro if called.
%    \begin{macrocode}
  \global\let\makeabstract\relax
  \global\let\makecopyright\relax
  \global\let\@getYear\@empty
  \global\let\@title\@empty
  \global\let\@author\@empty
  \global\let\@date\@empty
  \global\let\@discipline\@empty
  \global\let\@lastChapter\@empty
  \global\let\@director\@empty
  \global\let\title\relax
  \global\let\author\relax
  \global\let\date\relax
  \global\let\discipline\relax
  \global\let\lastChapter\relax
  \global\let\director\relax
}
%    \end{macrocode}
% \end{macro}
%
% \subsubsection{Acknowledgments Page}
% \begin{macro}{\makeacknowledgments}
% The |\makeacknowledgments| macro typesets the acknowledgments page.
% The acknowledgment starts a new unnumbered chapter and adds the chapter to the
% Table of Contents.
%    \begin{macrocode}
\newcommand{\makeacknowledgments}[1]{%
  \chapter*{Acknowledgments}
  \addcontentsline{toc}{chapter}{ACKNOWLEDGMENTS}
  #1\par
}
%    \end{macrocode}
% \end{macro}
%
% \subsection{Internal Commands for Preliminary Pages}
%
% \begin{macro}{\@getYear}
% The |\@getYear| macro strips off and returns the last four characters
% of the input argument. When a typical date is input to this macro,
% the macro evaluates to the four-digit expansion of the year. The
% |\fullexpandarg| ensures that the input argument is fully expanded before
% the characters are read and returned.
%    \begin{macrocode}
\newcommand\@getYear[1]{\fullexpandarg\StrRight{#1}{4}}
%    \end{macrocode}
% \end{macro}
%
% \begin{macro}{\@typesetCommittee}
% The lines for committee member signatures are typeset recursively one at a time.
% The recursive operation is handled by conditional command, |\IfSubStr|, checking
% if the remaining committee members string contains the |\\| character sequence.
% Note, that the first macro in each argument will be expanded once, as specified
% by the |\expandarg| macro, so macros that should not be expanded (i.e. |\\|) are
% prefixed with |\empty|, which expands to nothing.
%
% If the |\@committee| macro contains a |\\| sequence,
% the first committee member before the |\\| is saved
% into the macro |\@member|, and the |\@committee| macro
% is updated to be the string following the |\\| characters.
% The \DescribeMacro{\@typesetMember}|\@typesetMember| macro
% is used to make the signature line for the member,
% and the |\typesetCommittee| macro recursively typesets the remainder of the
% committee.
%
% When the |\@committee| macro no longer contains a |\\| sequence,
% the recursive algorithm has finished. The remaining committee members string
% is typeset as the last member and the |\@committee| string is cleared to ensure
% that no further processing is done on it.
%    \begin{macrocode}
\newcommand{\@typesetCommittee}{%
  \expandarg
  \IfSubStr{\@committee}{\empty\\}{%
    \StrCut{\@committee}{\empty\\}{\@member}{\@committee}
    \@typesetMember\@member
    \@typesetCommittee{\@committee}
  }{%
    \@typesetMember\@committee
    \global\let\@committee\@empty
  }
}
%    \end{macrocode}
% \end{macro}
%
%
% The |\@typesetMember| macro is used to draw the signature line and name-label
% for each committee member. Additionally, the first committee member should be
% designated as the chair. A conditional, \DescribeMacro{@ischair}|@ischair|,
% is declared globally and set to true. 
% After the first committee member's signature line has been typeset, the conditional
% is set false and remains so for the remainder of the committee. 
% The \DescribeMacro{\@chairDescriptor}|\@chairDescriptor|
% macro contains the text to be placed after the signature line of the chair member.
%    \begin{macrocode}
\newif\if@ischair
\@ischairtrue
\newcommand{\@chairDescriptor}{,~Chair}
%    \end{macrocode}
%
% \begin{macro}{\@typesetMember}
% The |\@typesetMember| macro takes one argument, \meta{member}.
% Since \meta{member} may contain a space at its beginning, the space
% is removed for consistent formatting.
%    \begin{macrocode}
\newcommand{\@typesetMember}[1]{%
  \expandarg
  \IfBeginWith{#1}{\space}{\StrBehind{#1}{\space}[#1]}{}
%    \end{macrocode}
%
% Next, the |@ischair| conditional is checked to determine if this signature line
% should be formatted for the chair member. If |@ischar| is true, the contents of
% the member string on the input before any commas (after which the degrees
% are typically listed) are saved in the |\@chair| macro.
% If the \DescribeMacro{\@director}|\@director| macro (used in the Abstract)
% has not already been explicitly defined,
% |\@director| is globally defined with the expansion of the |\@chair| macro,
% otherwise no action is taken.
% If the |@ischair| macro indicates that the current member is not the chair member
% the |\@chairDescriptor| is redefined as an empty macro.
%    \begin{macrocode}
  \if@ischair
    \StrBefore{#1}{,}[\@chair]
    \@ifundefined{@director}{\global\edef\@director{\@chair}}{}
  \else
    \renewcommand{\@chairDescriptor}{}
  \fi
%    \end{macrocode}
%
% The signature line is now ready to typeset. Two picas (24 points) of blank space
% are reserved for the physical signatures themselves.
% Then, a box is made to contain the line and the member's name with a width
% half the width of the printable area. A 0.4 point line is drawn across this box
% with the |\@chairDescriptor| label (possibly empty) added as a right-overlap
% field, allowing it to extend outside of the containing box.
%
% Negative vertical space of 10 points is added to ensure that the member's name
% is right against the signature line, and the member's name is typeset in yet
% another box to ensure that it remains on a single line, even if it is too long
% to be contained by the larger box.
%    \begin{macrocode}
  \vspace{2pc}
  \parbox{0.5\linewidth}{%
    \rule{\linewidth}{0.4pt}\rlap{\@chairDescriptor}\par
    \vspace{-10pt}\mbox{#1}
  }\par
%    \end{macrocode}
% Finally, the |@ischair| conditional is set false to ensure that no remaining
% member's signature lines are indicated as chair members.
%    \begin{macrocode}
  \@ischairfalse
}
%    \end{macrocode}
% \end{macro}
%
%
% \subsection{Formatting Titles within the Body Text}
%
% To format the section titles in the body of the paper, the
% package |titlesec| is used. The package option |nobottomtitles*| ensures that section
% titles are not allowed to appear near the end of a page. The space needed to allow a
% title to be typeset is approximately given by
% |\renewcommand{\bottomtitlespace}{|\meta{length}|}|.
%    \begin{macrocode}
\RequirePackage[nobottomtitles*]{titlesec}
%    \end{macrocode}
% 
% \begin{macro}{\@chapterFontStyle}
% The |\@chapterFontStyle| macro stores the formatting desired for the chapter
% title name. For the frontmatter (content at the beginning of the document 
% while the pages are numbered
% using Roman numerals), the chapter titles should be typeset in normal text.
% However, for the mainmatter (content in the main content sections with 
% Arabic numeral numbering), the 
% chapter label should be normal but the chapter title should be emboldened.
% This macro is initialized to bold the text as for the mainmatter,
% causing individual chapters to be typeset correctly without needing to call
% the |\mainmatter| command (doing so could cause problems with page
% numbering). The |\@chapterFontStyle| is redefined to make the
% chapter titles normal using the |\frontmatter| macro.
% Upon executing the
% |\mainmatter| macro, the |\@chapterFontStyle| macro will be redefined
% to format the title using |\bfseries|.
%    \begin{macrocode}
\newcommand{\@chapterFontStyle}{\bfseries}
%    \end{macrocode}
% \end{macro}
%
% \subsubsection{Chapters}
% Chapter titles in the body of the text are next formatted using the
% |\titleformat| command from the |titlesec| package. The macro
% redefines the |\chapter| command with the desired formatting.
% The optional argument, \meta{shape} = display, puts the section
% label in a separate paragraph from the section title.
% The \meta{format} argument makes the 
% font normal sized, centers the title, and ensures that a multi-line
% chapter title will be single-spaced, and adds extra vertical space.
% The |\capital@letterheight| macro saves the height of a capital
% letter, which is used in an expression to place the title at the very
% top of the printable page (it is move down an inch later).
% The \meta{label} argument formats the label text to be have an
% uppercase chapter name followed by the chapter number. 
% The macro |\chaptertitlename| is redefined from Chapter to
% Appendix by the |titlesec| package to ensure that appendices are
% labeled with the appropriate section name.
% The \meta{sep} argument places a one line space (|\baselineskip|)
% between the section number and the section title.
% The \meta{before-code} formats the title and makes the title
% uppercase. As describe above, the |\@chapterFontStyle| macro
% will either keep the text normally formatted or emboldened,
% depending on the section.
% Finally, |\titlespacing| is used to add an extra inch before the title
% (making the total margin 2 inches).
% Since the |\doublespacing| of |setspace| sets
% |\baselinestretch| = 1.665, an empty line has a width of
% 1/1.665 = |0.6\baselinestretch|.
% |\titlespacing| also adds 0.6 of a |\baseline| between the
% chapter title and any following content for an empty line.
%    \begin{macrocode}
\newlength{\capital@letterheight}
\titleformat{\chapter}[display]%
  {\settoheight{\capital@letterheight}{H}
  \normalfont\filcenter\singlespacing%
  \vspace{\dimexpr -2\baselineskip + \capital@letterheight \relax}}%
  {\MakeUppercase{\chaptertitlename} \thechapter}%
  {\baselineskip}%
  {\@chapterFontStyle\MakeUppercase}%
\titlespacing{\chapter}{0pt}{1in}{0.6\baselineskip}
%    \end{macrocode}
%
% \subsubsection{Sections}
% Formatting for the section title is achieved similarly.
% The optional argument, \meta{shape}~=~block, puts the section label
% and title in a block (paragraph) without additional formatting.
% The \meta{format} label contains the beginning of an environment---defined 
% in the |changepage| package---to shrink
% the margins by 0.5 inches on both sides to keep the section number inside
% the margins when section numbering is on. No check is made to disable
% this behavior when section numbering is off---the narrow margins will
% be maintained. The environment is turned off in the
% optional \meta{code-after} block.
% The \meta{label} argument prints the section number followed by a
% |\quad| space. There is 0 points of separation between the section
% number and the section title. No special format is applied before the
% section title is typeset.
%    \begin{macrocode}
\RequirePackage{changepage}
\titleformat{\section}[block]{%
    \begin{adjustwidth}{0.5in}{0.5in}
    \normalfont\bfseries\filcenter\singlespacing
  }{\@sectionnumbering}%
  {0pt}{}
  [\end{adjustwidth}]
%    \end{macrocode}
%
% \subsubsection{Subsections}
% The subsection formatting is very similar to the section format, except
% that it is italicized an the \meta{shape} argument is specified as
% hang, which typesets a hanging label.
% A ragged-right argument is also supplied to let full title lines 
% move onto the next line
% without hyphenating or spilling into the margin.
%    \begin{macrocode}
\titleformat{\subsection}[hang]%
  {\normalfont\itshape\singlespacing\raggedright}
  {\@subsectionnumbering}{0pt}{}
%    \end{macrocode}
%
% \subsubsection{Subsubsections}
% The subsubsection format is specified to be in-line with the text. Therefore,
% the \meta{shape} = |runin| was used to get a run-in title. Half a line of extra
% space is added before the subsubsection with the |\vspace| macro,
% by scaling the |\baselineskip| length.
% The \meta{label} is left empty to avoid typesetting the subsubsection label.
% No space is specified between the label (which does not exist) and the
% subsubsection title. The title is preceded by a horizontal space equal to a
% paragraph indentation and is followed by a colon before the text of the paragraph.
%    \begin{macrocode}
\titleformat{\subsubsection}[runin]%
  {\normalfont}{}{0pt}{}[:]
\titlespacing{\subsubsection}{\parindent}%
  {\subsubsection@stretch\baselineskip}{\wordsep}
%    \end{macrocode}
% 
%
% \subsection{Formatting the Table of Contents}
%
% To format the titles within the Table of Contents (\toc), the
% package |titletoc| is used. To include the Bibliography and Index
% in the \toc, the |tocbibind| package is used, with the optional argument
% specified not to include the \toc again and use chapter level headings if possible.
% Finally, the Bibliography heading is entered in all-caps to appear formatted correctly
% in the Table of Contents.
%    \begin{macrocode}
\RequirePackage{titletoc}
\RequirePackage[nottoc, chapter]{tocbibind}
\settocbibname{BIBLIOGRAPHY}
%    \end{macrocode}
%
% For all entries in the \toc, the right margin should be equal.
% The macro |\contentsmargin| reserves 2em of space at all
% \toc levels for the page number on the right hand side.
% The optional argument adds extra space for lines before the last
% to ensure that they do not go past the leader lines of the last row.
% Text on lines before the last cannot be within half and inch
% of the page numbers.
% The optional arguments adds extra space for these rows.
%    \begin{macrocode}
\contentsmargin[0.5in]{2em}
%    \end{macrocode}
%
% \begin{macro}{\@tocChapterName}
% The |\@tocChapterName| macro stores the name of the current
% section heading at the chapter level, either |\chaptername| or
% |\appendixname|. The macro is initialized as the |\chaptername|,
% but will be redefined later when the appendix environment begins.
% This macro ensures that Chapters and Appendices are given
% appropriate labels in the \toc.
%    \begin{macrocode}
\newcommand{\@tocChapterName}{\chaptername}
%    \end{macrocode}
% \end{macro}
%
% \begin{macro}{\leaderStyle}
% The |\leaderStyle| macro stores the formatting desired for the leader lines in the
% \toc. This macro is used by all levels of sections in the \toc to ensure that
% formatting is consistent. The command adds a small space between the text
% before and the leader line, then defines the leader line as dots with 3pts of separation.
%    \begin{macrocode}
\newcommand{\leaderStyle}{\,\titlerule*[3pt]{.}}
%    \end{macrocode}
% \end{macro}
%
%
% A heading, labeled \textit{Page}, is desired above the page
% numbers on the right of the \toc. Therefore, an entry is added
% to the table of contents, named \textit{Page},
% with horizontal space beforehand filled up.
% The extra vertical space of |0.6\baselineskip| that was added
% by the chapter title and the extra |\baselineskip| from the second
% |\par| command are removed with the |\vspace| command.
% The \toc is set to display headings from 4 levels:
% Chapters, Sections, Subsections, and Subsubsections.
%     \begin{macrocode}
\addtocontents{toc}{~\vspace{-1.6\baselineskip}\par\hfill Page\par}
\setcounter{tocdepth}{4}
%    \end{macrocode}
%
% Redefine the style for the table of contents. First clear the page to
% ensure that the page number for the table of contents is correct,
% then rename the table of contents and add a reference to it
% to the \toc file.
% Then, turn off hyphenation while typesetting the \toc.
% Finally, do the rest of the commands
% for typesetting the table of contents.
%    \begin{macrocode}
\let\oldtableofcontents\tableofcontents
\def\tableofcontents{%
  \clearpage
  \renewcommand\contentsname{Table of Contents}
  \addcontentsline{toc}{chapter}{TABLE OF CONTENTS}
  \oldtableofcontents
}
%    \end{macrocode}
%
%
% \subsubsection{Chapters}
% Chapter entries use the |\titlecontents| macro from the |titletoc|
% package for formatting. The chapter title has 0 inches of space
% between the margin and the beginning of the \toc entry.
% The \meta{above-code} argument ensures that the
% entry will be single-spaced if more than one line
% and sets the hanging indent for the chapter label to be equal to
% the length of a |\quad| (i.e., |1em|), since the horizontal
% spacing later on only applies to the first line.
% |\filright| from the |titlesec| package is used to keep from 
% justifying the \toc entry.
% The next two fields, \meta{numbered-format} and
% \meta{numberless-format} give the formatting for
% numbered and unnumbered sections respectively.
% Both of these specify that the section
% title should be uppercase as the last argument.
% The numbered version also pushes the
% chapter title (via the |\contentspush| macro)
% to the right to make space for and
% typeset the chapter name and number, including a colon.
% A |\quad| space is added after the chapter 
% number to match the hanging indent already specified.
% Note that the label is placed in an |\rlap| environment
% to make chapter labels for chapters greater than 10
% line up with the previous labels.
% Then, the leader-line and page number are typeset.
%    \begin{macrocode}
\newlength{\normalparindent}
\setlength{\normalparindent}{\parindent}
\titlecontents{chapter}[0in]{%
    \singlespacing\hangindent=1em\filright}%
  {\contentspush{%
      \MakeUppercase\@tocChapterName~%
      \rlap{\thecontentslabel:}\hphantom{0:}}\quad%
  }{}{\leaderStyle\contentspage}
%    \end{macrocode}
%
% In order to capitalize the chapter titles in the
% table of contents while maintaining
% compatibility with the |hyperref| package,
% the |\@chapter| macro is redefined to include
% the |\MakeUppercase| directive included in the .toc file itself.
% The |etoolbox| package is used. It contains a function,
% |\patchcmd|, which searches for a string inside a
% macro expansion and replaces it with a user-defined string.
% This allows for the command to be modified
% to include all caps letters.
% The command contains the text argument in two places;
% therefore, two replacements need to be made.
%    \begin{macrocode}
\RequirePackage{etoolbox}
\patchcmd{\@chapter}
  {{chapter}{\protect \numberline {\thechapter }#1}}
  {{chapter}{\protect \numberline {\thechapter }
      {\texorpdfstring{\MakeUppercase{#1}}{#1}}}}{}{}
\patchcmd{\@chapter}
  {{chapter}{#1}}
  {{chapter}
    {\texorpdfstring{\MakeUppercase{#1}}{#1}}}{}{}
%    \end{macrocode}
%
%
% \subsubsection{Sections}
% Similarly, the section headings are defined.
% They are given a left indentation
% of 0.5 inches and a negative vertical spacing of one line.
% Since the space between lines is |-1.2\baselineskip| under
% double-spacing, an adjustment is made of |-0.6\baselineskip|.
% If more that one line, titles should be single-spaced.
% |\filright| from the |\titlesec| package is used to keep from
% justifying the \toc entry.
% The section number should not be
% printed in the \toc, so both \meta{numbered-format}
% and \meta{numberless-format} are empty.
% The leader-line and page number are typeset in the last argument.
%    \begin{macrocode}
\titlecontents{section}[0.5in]%
  {\vspace{-0.6\baselineskip}\singlespacing\filright}{}{}%
  {\leaderStyle\contentspage}
%    \end{macrocode}
%
%
% \subsubsection{Subsections}
% The subsection has the same format as the Section title,
% except that it is indented 1 inch on the left and is typeset in italics.
% The page number is explicitly returned from being in italics
% to normal text.
%    \begin{macrocode}
\titlecontents{subsection}[1in]%
  {\vspace{-0.6\baselineskip}\singlespacing\itshape\filright}{}{}%
  {\leaderStyle\textnormal{\contentspage}}
%    \end{macrocode}
%
%
% \subsubsection{Subsubsections}
% The subsubsection has the same format as the Section title,
% except that it is indented 1.5 inches on the left.
%    \begin{macrocode}
\titlecontents{subsubsection}[1.5in]%
  {\vspace{-0.6\baselineskip}\singlespacing\filright}{}{}%
  {\leaderStyle\contentspage}
%    \end{macrocode}
%
%
% \subsubsection{List of Tables/Figures}
% The List of Figures and the List of Tables are unnumbered chapters
% and therefore need to be capitalized separately.
% The macros |\listfigurename| and |\listtablename| are
% renewed to include the all-caps code.
%    \begin{macrocode}
\let\oldlistfigurename\listfigurename
\renewcommand{\listfigurename}
  {\texorpdfstring
    {\MakeUppercase{\oldlistfigurename}}{\oldlistfigurename}}
\let\oldlisttablename\listtablename
\renewcommand{\listtablename}
  {\texorpdfstring
    {\MakeUppercase{\oldlisttablename}}{\oldlisttablename}}
%    \end{macrocode}
%
% As with the \toc, a heading, labeled \textit{Page},
% is desired above the page numbers on the right of the \lot and \lof.
% Therefore, an entry is added to the table of contents, named
% \textit{Page}, with horizontal space beforehand filled up and
% a vertical offset.
%     \begin{macrocode}
\addtocontents{lot}{~\vspace{-1.6\baselineskip}\par\hfill Page\par}
\addtocontents{lof}{~\vspace{-1.6\baselineskip}\par\hfill Page\par}
%    \end{macrocode}
%
% Define styles for the |table| and |figure| environments.
% Both should have single spacing and should not be justified.
% A configurable space is reserved for the table or figure label.
%    \begin{macrocode}
\titlecontents{table}[\contents@labelwidth]{%
    \singlespacing\filright}%
  {\contentslabel{\contents@labelwidth}}{}%
  {\leaderStyle\contentspage}
%    \end{macrocode}
%    \begin{macrocode}
\titlecontents{figure}[\contents@labelwidth]{%
    \singlespacing\filright}%
  {\contentslabel{\contents@labelwidth}}{}%
  {\leaderStyle\contentspage}
%    \end{macrocode}
%
% We also want to remove the extra space placed between
% chapters by default by the |\@chapter| macro.
%    \begin{macrocode}
\patchcmd{\@chapter}
  {\addtocontents{lot}{\protect\addvspace{10\p@}}}
  {{}}
  {\typeout{Successfully patched @chapter}}
  {\typeout{Failed to patch @chapter}}
\patchcmd{\@chapter}
  {\addtocontents{lof}{\protect\addvspace{10\p@}}}
  {{}}
  {\typeout{Successfully patched @chapter}}
  {\typeout{Failed to patch @chapter}}
%    \end{macrocode}
%
%
% \subsection{Appendices}
%
% To add appendices to the document, the
% package |appendix| is used. The |title| optional argument adds 
% \textit{Appendix} to the appendix number in the text, and is active by default.
%    \begin{macrocode}
\RequirePackage[title]{appendix}
%    \end{macrocode}
% 
% The appendices environment is redefined to include a redefinition of the
% |\@tocChapterName| macro, as described above. The |\appendices| macro, 
% used when beginning an |\appendices| environment, has an macro appended
% to its end using a |\let|-|\def| structure and a temporary macro.
% The |\@appendixmatter| macro contains formatting instructions for the appendices
% and is defined below.
%    \begin{macrocode}
\let\oldappendices\appendices
\def\appendices{\oldappendices\@appendixmatter}
%    \end{macrocode}
%
%
% \subsection{Paper Sectioning}
%
% In order to provide different formats for different sections of the paper, the macros
% |\frontmatter|, |\mainmatter|, and |\backmatter| are introduced.
%
% \begin{macro}{\frontmatter}
% Ensure a new page starts and reset page numbers with roman numerals.
%    \begin{macrocode}
\newcommand{\frontmatter}{%
  \clearpage\pagenumbering{roman}%
  \renewcommand{\@chapterFontStyle}{\normalfont}%
}
%    \end{macrocode}
% \end{macro}
%
% \begin{macro}{\mainmatter}
% Ensure a new page starts and reset page numbers with roman numerals.
% Also, set the |\@chapterFontStyle| macro to change the format of the chapter
% titles to be bold-face.
%    \begin{macrocode}
\newcommand{\mainmatter}{%
  \clearpage\pagenumbering{arabic}%
  \renewcommand{\@chapterFontStyle}{\bfseries}%
}
%    \end{macrocode}
% \end{macro}
%
% \begin{macro}{\backmatter}
% Set the |\@chapterFontStyle| macro to change the format of the chapter
% titles to be regular font.
%    \begin{macrocode}
\newcommand{\backmatter}{%
  \renewcommand{\@chapterFontStyle}{\normalfont}%
}
%    \end{macrocode}
% \end{macro}
%
% \begin{macro}{\@appendixmatter}
% The |\@appendixmatter| macro is appended to the appendix environment and 
% sets the |\@chapterFontStyle| macro to change the format of the chapter
% titles to be bold-face, like the mainmatter.
% The |\@appendixmatter| macro also inserts a command into the generated
% .toc file to give the \toc a new value for the |\@tocChapterName| macro.
% The |\protect| macros ensure that the following macros are added to the .toc
% file as-is without expansion.
%    \begin{macrocode}
\newcommand{\@appendixmatter}{%
  \renewcommand{\@chapterFontStyle}{\bfseries}
  \addtocontents{toc}%
  {\protect\renewcommand*{\protect\@tocChapterName}%
    {\protect\appendixname}}
}
%    \end{macrocode}
% \end{macro}
%
% \subsection{Environments}
% \subsubsection{Tables}
% Formal table formatting is used from the |booktabs| package. Additionally,
% the |longtable| package is used for tables spanning multiple pages. Then,
% redefine the default width of the caption field for the |longtable| environment.
%    \begin{macrocode}
\RequirePackage{booktabs}
\RequirePackage{longtable}
\setlength{\LTcapwidth}{\textwidth}
%    \end{macrocode}
% Define a new macro for a single thin line above the |longtable| entry for text
% continued on the next page. Also, define a macro to set the desired spacing
% format for text within cells. The redefinition of |\arraystretch| further adjusts
% the vertical spacing between adjacent cells.
%    \begin{macrocode}
\newcommand{\continuerule}{
  \specialrule{\lightrulewidth}{\aboverulesep}{\belowbottomsep}
}
\newcommand{\settablespacingformat}{
  \onehalfspacing
  \renewcommand{\arraystretch}{1.2}
}
%    \end{macrocode}
% Require the package |xargs| to create an environment with non-standard
% optional arguments. Then define the |thesistable| environment, with the
% first and third arguments specified as optional. These arguments will have a default
% value that expands to the |\relax| macro.
% Several macros are defined, that when expanded, will evaluate to the same thing if
% the optional arguments were not provided.
%    \begin{macrocode}
\RequirePackage{xargs}
\newenvironmentx{thesistable}[6][1=\relax, 3=\relax]{
  \def\given@placement{#1}
  \def\default@placement{\relax}
%    \end{macrocode}
% A macro for the figure's caption is created containing only the \meta{caption}
% argument if the \meta{lofCaption} argument was not provided. Otherwise, the
% optional argument is passed along to the |\caption| command.
%    \begin{macrocode}
  \ifx\relax#3
    \def\table@caption{\caption{#4}}
  \else
    \def\table@caption{\caption[#3]{#4}}
  \fi
%    \end{macrocode}
% The type of table requested is checked based on the presence or absence of the 
% first optional command. If the float placement argument was not provided, start
% a |longtable| environment with specified width, caption, and label.
%    \begin{macrocode}
  \ifx\given@placement\default@placement
    \typeout{Typesetting environment thesistable as a longtable.}
    \settablespacingformat
    \begin{longtable}[c]{#2}
    \table@caption\label{#5} \\
%    \end{macrocode}
% Set the formatting and content for the header row on the first page the table appears.
%    \begin{macrocode}
    \toprule
    #6 \\
    \midrule \endfirsthead
%    \end{macrocode}
% Set the formatting and content for the header row on any subsequent pages on which
% the table appears. These headers should contain the table number with a note that
% the table is being continued from a previous page. This text is placed within another
% |\caption| command with an empty optional argument to ensure that this second caption
% is not listed in the List of Tables. The table number is references itself via the |\nameref|
% macro and the fifth argument, \meta{label}.
%    \begin{macrocode}
    \caption[]{\nameref{#5} 
        \hfill\mbox{ -- \textit{Continued from previous page}}}\\ [-1ex]
    \toprule
    #6 \\
    \midrule \endhead
%    \end{macrocode}
% Footers for any page but the last one contain a note that the table is continued on a subsequent
% page. These rows contain a lighter rule than the |\bottomrule| that ends the table.
%    \begin{macrocode}
    \continuerule
    \multicolumn{\LT@cols}{r}{\textit{Continued on next page}} \\
    \endfoot
%    \end{macrocode}
% Finally, the |longtable| format is finished by specifying the format for the final row of the table.
%    \begin{macrocode}
    \bottomrule \endlastfoot
%    \end{macrocode}
% After completing the header information for the |longtable| environment, the format for the
% floating table must also be specified. The |\else| command completes the if statement above
% that checked the existence of the first optional argument.
%    \begin{macrocode}
  \else
    \typeout{Typesetting environment thesistable as a tabular.}
    \begin{table}[#1]\centering
    \settablespacingformat
    \table@caption\label{#5}
    \vspace{\baselineskip}
    \begin{tabular}{#2}
%    \end{macrocode}
% Define the formatting for the header row. Then finish the if statement and the first argument
% that specifies the beginning of the |thesistable| environment.
%    \begin{macrocode}    
    \toprule
    #6 \tabularnewline
    \midrule
  \fi
}
%    \end{macrocode}
% The second argument of the |thesistable| environment declares how to end the environment.
% An if statement, similar to the first part, is used to either wrap up the |longtable| or the
% |tabular| and |table| depending on which one was created.
%    \begin{macrocode}
{
  \ifx\given@placement\default@placement
    \end{longtable}
  \else
    \bottomrule
    \end{tabular}
    \end{table}
  \fi
}
%    \end{macrocode}
%
%
% \subsection{Other Spacing}
% Using the double-spaced formatting causes the spacing around equations to be messed up.
% The |\normalsize| macro is redefined to update the equation spacing. The skip spaces
% before and after equations are defined with 3pts of stretchable length. If needed, \LaTeX{}
% will be able to add up to 3pt separation between the body text and the equation to space
% large equations well (e.g., those with integrals or matrices).
%    \begin{macrocode}
\g@addto@macro\normalsize{%
  \setlength\abovedisplayskip{0pt plus 3pt}
  \setlength\belowdisplayskip{0pt plus 3pt}
  \setlength\abovedisplayshortskip{0pt plus 3pt}
  \setlength\belowdisplayshortskip{0pt plus 3pt}
}
%    \end{macrocode}
% Additionally, the |\jot| and |\arraystrech| values are redefined to adjust the spacing
% between equations and the spacing inside matrices, respectively.
%    \begin{macrocode}
\setlength{\jot}{-1ex}
\renewcommand*\arraystretch{.6}
%    \end{macrocode}
%
%
% \subsection{Footnotes}
% To help with formatting for footnotes, the |footmisc| package is used.
% This package is loaded with several options that define the base footnote formatting.
% The option |multiple| places a comma between two footnotes placed at the same spot.
% The option |hang| formats the footnotes as hanging paragraphs.
% The option |perpage| resets the numbering of the footnotes on each page.
% (Note that pdfLaTeX may need to be run twice to correctly place and number
% all footnotes on the correct pages.)
% The option |symbol*| uses a set of symbols instead of numerals to number the footnotes.
% The |bottom| option ensures that footnotes are typeset
% below floats.
% The |stable| option keeps |footmisc| from
% including footnotes that appear in section titles in the \toc.
%    \begin{macrocode}
\RequirePackage[multiple,hang,perpage,symbol*,bottom,stable]{footmisc}
%    \end{macrocode}
%
% Footnotes are made the same size as the rest of the document.
% The left margin is decreased to 1em to place the footnote closer to the symbol.
% The original footnote separation is saved for a future calculation, and the footnote separation
% is increased to the value |\baselineskip| to place double spacing between footnotes.
% Note, if the footnote font is less than |\normalsize|, the |\footnotesep| should similarly be
% decreased, as it seems to be set on the |\baselineskip| of the main text.
% The |\footnotesep| should be specified as greater or equal to the default value of 8.4 points.
%    \begin{macrocode}
\let\footnotesize\normalsize
\setlength{\footnotemargin}{1em}
\newlength{\@defaultfootsep}
\setlength{\@defaultfootsep}{\footnotesep}
\setlength{\footnotesep}{\baselineskip}
%    \end{macrocode}
%
% \begin{macro}{\@footnoteheadspace}
% The absolute amount of space between the footnote text and the rule above is tuned with the macro
% |\footnoteheadspace|. Both the default footnote separation |\skip\footins| and 
% |0.4\baselineskip| look nice.
%    \begin{macrocode}
\newlength{\@footnoteheadspace}
\setlength{\@footnoteheadspace}{\skip\footins}
%    \end{macrocode}
% \end{macro}
%
% \begin{macro}{\@footnoteheadspaceadjust}
% An expression is saved for the adjustment factor for the kerning of the footnote rule.
% Values were determined from trial-and error and linear relationships, and when simplified, resulted in
% unit multiples of several macros. With |\@footnoteheadspace| set to zero, the footnote rule should be directly
% on top of the footnote text. The formula seems to depend on the difference between the footnote separation
% and the default separation, because setting the footnote separation below the default value seems to have
% no effect on the spacing of the typeset output. The exact value at which changes no longer make differences
% may need to be further tuned for exact placement of the rule.
%    \begin{macrocode}
\newlength{\@footnoteheadspaceadjust}
\setlength{\@footnoteheadspaceadjust}{\dimexpr \footnotesep - \@defaultfootsep - \@footnoteheadspace}
%    \end{macrocode}
% \end{macro}
%
% \begin{macro}{\@footnoteruleheight}
% The thickness of the footnote rule is defined in the |\@footnoteruleheight| macro.
%    \begin{macrocode}
\newlength{\@footnoteruleheight}
\setlength{\@footnoteruleheight}{0.4pt}
%    \end{macrocode}
% \end{macro}
% 
% \begin{macro}{\footnoterule}
% Create the footnote rule, adjusting the kerning beforehand to place the rule at the correct spot and 
% adjusting the kerning afterwards to return to the previous spot while adjusting for the rule thickness.
%    \begin{macrocode}
\renewcommand{\footnoterule}{%
  \kern \@footnoteheadspaceadjust 
  \hrule width \textwidth height \@footnoteruleheight
  \kern\dimexpr -\@footnoteheadspaceadjust-\@footnoteruleheight \relax
}
%    \end{macrocode}
% \end{macro}
%
%
% \Finale
%
%
% \typeout{****************************************************}
% \typeout{*}
% \typeout{* To finish the installation you have to move the}
% \typeout{* following file into a directory searched by TeX:}
% \typeout{*}
% \typeout{* \space\space letuThesis.sty}
% \typeout{*}
% \typeout{* To produce the documentation run the file}
% \typeout{* letuThesis.dtx through LaTeX.}
% \typeout{*}
% \typeout{* Happy TeXing!}
% \typeout{*}
% \typeout{****************************************************}
\endinput

%% Create changelog and index using:
%% makeindex -s gind.ist -o letuThesis.ind letuThesis.idx && makeindex -s gglo.ist -o letuThesis.gls letuThesis.glo


